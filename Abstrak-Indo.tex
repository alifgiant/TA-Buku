\chapter*{Abstrak}
%--Overview-- \\
Penyakit jantung (Cardiovascular Diseases, CVDs) merupakan penyakit yang dapat menyerang siapa saja, terjadi kapan saja dan dimana saja. Terdapat banyak produk di pasaran yang dapat melakukan \textit{monitoring} jantung sekaligus merekam aktivitas jantung penggunanya. Rekam jantung diperlukan oleh dokter jantung untuk melakukan analisis penyakit dan merancang metode pengobatan. Salah satu jenis CVDs yang dapat diidentifikasi dari rekam jantung ialah Aritmia. Menurut literatur, Aritmia adalah ritme detak jantung yang tidak teratur. Beberapa penelitian sebelumya telah berhasil mendeteksi Aritmia secara otomatis.	
%--Problem-- \\
Namun produk tersebut tidak dapat melakukan \textit{monitoring} secara terus menerus dan tidak dapat melakukan analisis otomatis. Analisis harus dilakukan secara manual oleh seorang dokter jantung. Di lain pihak, seorang dokter jantung ditargetkan mengawasi sangat banyak pasien (sekitar 100 ribu orang). Dalam beberapa kasus, pasien juga memerlukan pengawasan terus menerus. Internet of Things(IoT) telah terbukti handal untuk melakukan monitoring banyak pasien secara terus menerus. Namun implementasi IoT untuk monitoring tersebut masih kurang efisien, seperti tidak adanya notifikasi kepada dokter secara \textit{real time}. Algoritma yang diusulkan pada penelitian sebelumnya pun belum dirancang untuk memaksimalkan kecepatan eksekusi agar dapat memproses banyak pasien secara bersamaan.
%--Objective-- \\
Oleh karena itu tugas akhir ini merancang arsitektur sistem IoT dan menerapkan modifikasi algoritma yang diusulkan penelitian sebelumnya untuk menyelesaikan masalah diatas.
%--Methodology-- \\
Rancangan arsitektur memanfaatkan MQTT sebagai protokol komunikasi jaringan yang dirancang agar dapat memproses banyak \textit{sensor} dan banyak \textit{viewer} sekaligus. Algoritma yang diterapkan pada arsitektur ialah modifikasi algoritma usulan Pan-Tompkin(1985), Tsipouras(2005) dan Kalidas-Tamil(2016).
Pan-Tomkins dan Kalidas-Tamil mengusulkan metode untuk melakukan \textit{preprocessing} dan \textit{processing} terhadap hasil baca sensor. Tsipouras mengusulkan metode \textit{rule based classification} untuk mengklasfikasikan aritmia dengan menggunakan fitur \textit{peak} (titik R atau titik Systolic) pada rekaman jantung.
%--Outcome-- \\
Berdasarkan pengujian, performa arsitektur sistem yang diusulkan pada tugas akhir ini dinilai baik, yaitu dapat melayani maksimum xx sensor dengan rata-rata delay xx ms dan tingkat akurasi xx\% untuk deteksi detak walau deteksi aritmia memiliki perfoma yang buruk hanya xx\%.
  
\vspace{0.5 cm}
\begin{flushleft}
{\textbf{Kata Kunci:} Monitoring Jantung, Aritmia, IoT, MQTT.}
\end{flushleft}