\chapter*{Abstrak}
%--Overview-- \\
Penyakit jantung (Cardiovascular Diseases, CVDs) merupakan penyakit yang dapat menyerang siapa saja, terjadi kapan saja dan dimana saja. Terdapat banyak produk di pasaran yang dapat melakukan \textit{monitoring} jantung sekaligus merekam aktivitas jantung penggunanya. Rekam jantung diperlukan oleh dokter jantung untuk melakukan analisis penyakit dan merancang metode pengobatan. Salah satu jenis CVDs yang dapat diidentifikasi dari rekam jantung ialah Aritmia. Aritmia adalah ritme detak jantung yang tidak teratur. Beberapa penelitian sebelumya telah berhasil mendeteksi Aritmia secara otomatis.	
%--Problem-- \\
Namun produk tersebut tidak dapat melakukan \textit{monitoring} secara terus menerus dan tidak dapat melakukan analisis otomatis. Analisis harus dilakukan secara manual oleh seorang dokter jantung. Di lain pihak, seorang dokter jantung ditargetkan mengawasi sangat banyak pasien. Dalam beberapa kasus, pasien juga memerlukan pengawasan terus menerus. Internet of Things(IoT) telah terbukti handal untuk melakukan monitoring banyak pasien secara terus menerus. Namun implementasi IoT untuk monitoring tersebut masih kurang efisien, seperti tidak adanya notifikasi kepada dokter secara \textit{real time}. Algoritma yang diusulkan pada penelitian sebelumnya pun belum dirancang untuk memaksimalkan kecepatan eksekusi agar dapat memproses banyak pasien secara bersamaan.
%--Objective-- \\
Oleh karena itu, pada tugas akhir ini dirancang sebuah arsitektur sistem IoT yang menerapkan modifikasi algoritma atas usulan penelitian sebelumnya untuk menyelesaikan masalah diatas.
%--Methodology-- \\
Rancangan memanfaatkan MQTT sebagai protokol komunikasi jaringan agar dapat memproses banyak \textit{sensor} dan banyak \textit{dashboard} sekaligus. Algoritma yang diterapkan pada arsitektur ialah modifikasi algoritma usulan Pan-Tompkin(1985), Tsipouras(2005) dan Kalidas-Tamil(2016). Pan-Tomkins dan Kalidas-Tamil mengusulkan metode untuk melakukan \textit{preprocessing} dan \textit{processing} terhadap hasil baca sensor. Tsipouras mengusulkan metode \textit{rule based classification} untuk mengklasfikasikan aritmia dengan menggunakan fitur \textit{peak} pada rekaman jantung.
%--Outcome-- \\
Berdasarkan hasil pengujian, performa arsitektur sistem yang diusulkan pada tugas akhir ini dinilai baik. Arsitektur dapat melayani maksimum \sensor \textit{devices} dengan rata-rata delay \delay dan tingkat akurasi \accuracy.
  
\vspace{0.5 cm}
\begin{flushleft}
{\textbf{Kata Kunci:} Monitoring Jantung, Aritmia, IoT, MQTT.}
\end{flushleft}