\chapter*{Abstrak}

Pada tahun 2015 diperkirakan sebanyak 17,7 juta kematian disebabkan oleh penyakit kardiovaskuler (penyakit jantung).  Pada penyakit jantung seringkali ditandai dengan munculnya pola tidak beratur pada detak jantung seseorang. Pola ini dikenal dengan istilah Aritmia. Mengetahui terjadinya pola ini dapat menyelamatkan banyak nyawa. Namun keahlian untuk menganalisis pola detak jantung hanya dimiliki oleh mereka yang telah mengemban pendidikan kesehatan seperti dokter jantung. Pada beberapa penelitian sebelumnya telah dikembangkan berbagai metode pengukuran detak jantung non-invasive dan telah dikembangkan pula metode mendeteksi terjadinya Aritmia. Namun sistem yang ada tidak dapat memberikan peringatan dini ketika Aritmia terjadi. Tugas akhir ini mengusulkan sebuah rancangan sistem pemberian peringatan dini kepada orang terdekat dan atau dokter tentang terjadinya Aritmia.
  
\vspace{0.5 cm}
\begin{flushleft}
{\textbf{Kata Kunci:} Penyakit Jantung, Aritmia, Peringatan Dini.}
\end{flushleft}