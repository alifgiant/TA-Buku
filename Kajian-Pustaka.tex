\chapter{Kajian Pustaka}

\section{ECG dan PPG}
Terdapat 2 jenis sensor yang umum digunakan untuk melakukan monitoring jantung, yaitu \textit{Electrocardiogram} (ECG) dan \textit{Photoplethysmogram} (PPG). Kedua jenis sensor ini menjadi pilihan utama dalam monitoring jantung karena keduanya mengusung konsep \textit{non-invasive}. Sensor non-invasive memungkinkan melakukan pengambilan data tubuh tanpa perlu melukai/menusuk bagian tubuh tertentu. Walaupun demikian kedua jenis sensor ini memiliki kelebihan dan kekurangan untuk digunakan. Secara umum ECG akan menghasilkan pengukuran lebih akurat dari pada PPG. Namun PPG lebih nyaman digunakan dalam jangka panjang dari pada ECG.

\begin{equation}\label{nama-rumus}
    \int_0^1 \frac{f(x)}{g(x)}\ {\rm dx}=\sin x
\end{equation}

Rumus (\ref{nama-rumus}) merupakan contoh persamaan matematika. persamaan matematika diatas diberi nama \textbackslash label\{nama-rumus\}.

\begin{figure}[h!]
    \centering
    \includegraphics[scale=0.3]{Tel-U-Logo.png}
    \caption{Caption}
    \label{fig:my_label}
\end{figure}
\subsection{Cara memanggil pustaka}
Contoh pustaka prosiding \cite{doyen2014explicit}, jurnal \cite{gunawan2015hydrostatic} dan buku \cite{toro2013riemann}. Atau dapat juga mengguanakan dua pustaka atau lebih dalam \cite{gunawan2015hydrostatic,toro2013riemann}.