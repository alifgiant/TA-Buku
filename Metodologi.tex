\chapter{Metodologi dan Desain Sistem}
\section{Metodologi Penelitian}
Metode yang digunakan dalam menyelesaikan tugas akhir ini:
\begin{enumerate}
	\item \textbf{Studi literatur} \\
	Pada tahap ini penulis mengumpulkan literatur seperti buku, artikel dan \textit{paper} yang berguna menjadi landasan informasi pada penelitian. Hasil tahap ini ialah fakta dan teori serta masalah yang dihadapi.
	\item \textbf{Perancangan Sistem} \\
	Pada tahap ini penulis memilah masalah yang dapat diselesaikan berdasarkan fakta dan teori yang telah dikumpulkan. Hasil tahap ini ialah rancangan sistem yang diajukan sebagai solusi.
	\item \textbf{Persiapan Data Uji} \\
	Pada tahap ini penulis mempersiapkan data yang telah tervalidasi kebenarannya untuk dijadikan input pengujian. Hasil tahap ini ialah dataset yang telah dianotasi.
	\item \textbf{Implementasi} \\
	Pada tahap ini penulis menerapkan rancangan sistem baik yang berupa \textit{software} maupun \textit{hardware}. Hasil tahap ini ialah \textit{software} dan \textit{hardware} yang dapat berjalan tanpa masalah.
	\item \textbf{Pengujian dan Analisis} \\
	Pada tahap ini penulis melakukan pengujian terhadap sistem yang dibangun menggunakan data uji dan parameter pengujian. Jika ditemukan ada masalah teknis ataupun kemungkinan melakukan peningkatan performansi maka penulis akan kembali ke tahap implementasi. Hasil tahap ini ialah \textit{software} dan \textit{hardware} dengan konfigurasi terbaik yang ditemukan.
	\item \textbf{Penyusunan Laporan} \\
	Pada tahap ini penulis melakukan penulisan laporan hasil akhir dari tugas akhir. Hasil dari tahap ini berupa buku tugas akhir dan jurnal penelitian.
\end{enumerate}

%Definisikan bentuk dan warna
\usetikzlibrary{positioning}
\tikzstyle{cloud} = [draw, ellipse,fill=red!20, minimum height=2em]
\tikzstyle{startstop} = [rectangle, rounded corners, minimum width=3cm, minimum height=1cm,text centered, draw=black, fill=red!30]
\tikzstyle{io} = [trapezium, trapezium left angle=70, trapezium right angle=110, minimum width=3cm, minimum height=1cm, text centered, draw=black, fill=blue!30]
\tikzstyle{process} = [rectangle, minimum width=3cm, minimum height=1cm, text centered, minimum width=3cm, draw=black, fill=orange!30]
\tikzstyle{decision} = [diamond, minimum width=3cm, minimum height=1cm, text centered, draw=black, fill=green!30]
\tikzstyle{arrow} = [thick,->,>=stealth]
\tikzstyle{db} = [cylinder, draw, shape border rotate=90, minimum height=2em, minimum width=2em, fill=red!30]

\begin{figure}[h!]
    \centering
    %Mulai menggambar Flowchart
\begin{tikzpicture}[node distance=2cm]
\node (start) [cloud] {Start};
\node (studi) [process, below of=start] {Studi Literatur};
\node (design) [process, below of=studi] {Perancangan Sistem};
\node (data) [process, below of=design] {Persiapan Data Uji};
\node (implementasi) [process, below of=data] {Implementasi};
\node (pengujian) [process, below of=implementasi] {Pengujian dan Analisis};
\node (yesno) [decision, below of=pengujian] {Perbaikan?};
\node (report) [process, below of=yesno] {Penyusunan Laporan};
\node (stop) [cloud, below of=report] {Stop};
\draw [arrow] (start) -- (studi);
\draw [arrow] (studi) -- (design);
\draw [arrow] (design) -- (data);
\draw [arrow] (data) -- (implementasi);
\draw [arrow] (implementasi) -- (pengujian);
\draw [arrow] (pengujian) -- (yesno);
\draw [arrow] (yesno.west) -- ++(-40pt,0pt) |- (implementasi.west);
\draw [arrow] (yesno) -- (report);
\draw [arrow] (report) -- (stop);
\end{tikzpicture}
    \caption{Caption flowchart}
    \label{figflow}
\end{figure}

\section{Algoritma}
 Atau dalam bentuk algoritma seperti contoh pada Algoritma \ref{Algo:FVDM} berikut ini:
 

\begin{algorithm}
 \begin{algorithmic}[1]
    \Procedure{FVDM}{$Tfinal, \Delta t$}
    \State \text{Start}
	\State \textbf{For }$n=1:N$\textbf{ do} \Comment{Pemberian nilai awal}
	\State \hspace{0.5cm} Input nilai $x[n]$
	\State \hspace{0.5cm} Input nilai $v[n]$
	\State \textbf{EndFor}
	\State \text{time=0}
	\While{$time < Tfinal$}
	\State \hspace{0.5cm} $time=time +\Delta t$
	\State \hspace{0.5cm} Hitung jarak bamper menggunakan rumus  untuk $n=2,\cdots,N$ 
	\State \hspace{0.5cm} \textbf{If}( $S(n) \leq 0 m)$ \textbf{then return End If}.
	\State \hspace{0.5cm} Tentukan $\lambda$ menggunakan.
	\State \hspace{0.5cm} Hitung kecepatan optimal $v_o(t)$ menggunakan.
	\State \hspace{0.5cm} Hitung percepatan $a_n(time)$ menggunakan .
	\State \hspace{0.5cm} Hitung kecepatan baru dengan $v_n(time)=v_n(time-\Delta t) + a_n(time) \Delta t$.
	\State \hspace{0.5cm} Hitung posisi baru dengan $x_n(time)=x_n(time-\Delta t) + v_n(time) \Delta t$.
	\State \hspace{0.5cm} \textbf{If}( $\Delta v \leq 10^{-5} \&\& a_n(time)\leq 10^{-5})$ \textbf{then}  
    \State \hspace{0.5cm} \hspace{0.5cm} \text{OUTPUT }Cetak hasil data $a_n, v_n, x_n$.
	\State \hspace{0.5cm} \hspace{0.5cm} \textbf{return}.
	\State \hspace{0.5cm} \textbf{End If}.
	\EndWhile
	\State \text{End}
 \EndProcedure
 \end{algorithmic}
 \caption{Prosedur simulasi dinamika lalu lintas menggunakan FVDM.}\label{Algo:FVDM}
\end{algorithm}