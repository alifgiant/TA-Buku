\chapter*{Abstract}
%--Overview-- \\
Heart disease (Cardiovascular Diseases, CVD) is a disease that can affect anyone, happens anytime and anywhere. There are many products on the market that can perform cardiac monitoring while recording cardiac activity. Cardiac records are required by cardiologists to perform disease analysis and to design treatment methods. One type of identifiable CVDs of cardiac arrest is arrhythmias. The arrhythmia is an irregular heartbeat rhythm. Several previous studies have successfully detected arrhythmia automatically. 
%--Problem-- \\
However, the product can not monitor continuously and can not perform an automated analysis. The analysis has to be done manually by a cardiologist. On the other hand, a cardiologist is targeted to monitor very many patients. In some cases, patients also require ongoing supervision. Internet of Things (IoT) has proven to be reliable for monitoring many patients on a continuous basis. But the implementation of IoT for monitoring is still less efficient, such as the absence of notification to the doctor in real time. Also, the algorithm proposed in previous studies has not been designed to maximize execution speed in order to process many patients simultaneously. 
%--Objective-- \\
Therefore, in this final project, writer designed an IoT system architecture that applies algorithm modification on the previous research proposal to solve the problem. 
%--Methodology-- \\
The architecture design utilizes MQTT as a network communication protocol in order to process multiple sensors and multiple dashboards at once. The algorithm applied to the architecture is the modification of algorithm proposed by Pan and Tompkin (1985), Tsipouras (2005) and Kalidas-Tamil (2016). Pan-Tomkins and Kalidas-Tamil proposed methods for preprocessing and processing of sensor readings. Tsipouras proposed a rule-based classification method to clarify arrhythmias using peak features on heart recordings. 
%--Outcome-- \\
Based on the test results, the performance of the proposed system architecture in this final project is considered as good. It can serve a maximum of \sensor devices with an average delay of \delay and a accuracy rate of \accuracy.

\vspace{0.5 cm}
\begin{flushleft}
{\textbf{Keywords:} Heart Monitoring, Arrhythmia, IoT, MQTT.}
\end{flushleft}