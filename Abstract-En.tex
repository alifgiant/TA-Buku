\chapter*{Abstract}

Heart disease (Cardiovascular Diseases, CVDs) is a disease that can affect anyone, happens anytime and anywhere. There are many products on the market that can perform heart monitoring while recording cardiac (heart) activity. Cardiac records are required by cardiologists to perform disease analysis and to design treatment methods. One type of CVDs that can be identified from the cardiac record is the arrhythmia. According to the literature, the arrhythmia is an irregular heartbeat rhythm. Several previous studies have successfully detected arrhythmia automatically. However, the product can not monitor continuously and can not perform an automated analysis. The analysis has to be done manually by a cardiologist. On the other hand, a cardiologist is targeted to monitor very many patients (about 100 thousand people). This makes a doctor impossible to monitor anyone continuously. Though by applying the concept of Internet of Things (IoT) and algorithms proposed in previous research, the system can work continuously and provide warnings about arrhythmia to a cardiologist who is connected in real time. Therefore this final project designs the IoT system architecture and applies the proposed algorithm of previous research to solve the above problem. The architecture design utilizes MQTT as a network communication protocol designed to process multiple sensors and multiple viewers at once. Algorithm applied to the architecture is the modification algorithm proposed Pan-Tompkin (1985), Deshmane (2009) and Tsipouras (2005). Pan-Tomkins and Deshmane proposed a method for preprocessing and processing the sensor readings data. Tsipouras proposed a rule-based classification method to classify arrhythmias using the peak feature (point R or Systolic point) of cardiac record. Based on testing, the performance of the proposed system architecture in this final project is considered good, that is, it can serve a maximum of xxx sensor with average delay xxx ms and xx\% accuracy level for detection of transient detection although arrhythmia has poor performance with accuracy only xx\%.

\vspace{0.5 cm}
\begin{flushleft}
{\textbf{Keywords:} Heart Monitoring, Arrhythmia, IoT, MQTT.}
\end{flushleft}